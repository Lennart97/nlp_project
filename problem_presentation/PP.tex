%----------------------------------------------------------------------------------------
%	PACKAGES AND THEMES
%----------------------------------------------------------------------------------------
\documentclass[aspectratio=169,xcolor=dvipsnames]{beamer}
\usetheme{Simple}

\usepackage{hyperref}
\usepackage{graphicx} % Allows including images
%\usepackage[export]{adjustbox} %throws error
\usepackage{booktabs} % Allows the use of \toprule, \midrule and \bottomrule in tables
\setbeamersize{text margin left=12mm, text margin right=12mm} 

\addtobeamertemplate{footline}{%
  \setlength\unitlength{1ex}%
  \begin{picture}(0,0) 
    % \put{} defines the position of the frame
    \put(134,83.6){\makebox(0,0)[bl]{
    \includegraphics[scale=0.4]{images/uni_logo_weiss.eps}
    }}%
  \end{picture}%
}{}

%----------------------------------------------------------------------------------------
%	TITLE PAGE
%----------------------------------------------------------------------------------------

% The title
\title[Determination of political views]{Determination of political views}
\subtitle{using NLP on social media postings}

\author[Lennart Hallenberger, Florian Siepe]{Lennart Hallenberger, Florian Siepe}
\institute[] % Your institution may be shorthand to save space
{
    % Your institution for the title page
    Department Mathematics and Computer Science \\
    Philipps University of Marburg
    \vskip 3pt
}
\date{\today} % Date, can be changed to a custom date


%----------------------------------------------------------------------------------------
%	PRESENTATION SLIDES
%----------------------------------------------------------------------------------------

\begin{document}

\begin{frame}[plain]
    % Print the title page as the first slide
    \titlepage
    \begin{figure}[h]
        \includegraphics[scale=0.5]{images/uni_logo_schwarz.eps}
    \end{figure}
\end{frame}

\begin{frame}{Outline}
    % Throughout your presentation, if you choose to use \section{} and \subsection{} commands, these will automatically be printed on this slide as an overview of your presentation
    \tableofcontents
\end{frame}
%------------------------------------------------
\section{Motivation}
%------------------------------------------------

\begin{frame}{Motivation}
    \begin{itemize}
        \item Lack of pre-classified data of political viewpoints
        \begin{itemize}
            \item and their tendency to be outdated
        \end{itemize}
        \item Rising importance of social media for politicians
        \item Gaining insights on political topics and their distribution
        \item Improving election projections
    \end{itemize}
\end{frame}

%------------------------------------------------
\section{Goal}
%------------------------------------------------

\begin{frame}{Goal}

    \begin{block}{Main objective}
        Classify users based on their social media postings to match them with a political party.
    \end{block}
    
    \begin{block}{Verification}
        Verify precision and recall of our approach.
    \end{block}
\end{frame}

%------------------------------------------------
\section{Idea}
%------------------------------------------------

\begin{frame}{Idea}

    \begin{block}{Concept}
        Creating a reusable pipeline instead of a static dataset.
    \end{block}

    \begin{block}{Approach}
        \begin{itemize}
            \item Use posts from known politicians with their political party as label
            \item Extract relevant features which distinguish the labels
        \end{itemize}
    \end{block}
    
\end{frame}

%------------------------------------------------
\section{Implementation}
%------------------------------------------------

\begin{frame}{Implementation}
    \begin{example}
        \begin{itemize}
            \item Create an equal set of republican and democratic politicians from the US
            \item Gather an equal amount of current tweets 
            \item Classify tweets with the political party of their author
            \item Extract distinguishing features between labels
            \item Apply model to politicians who were not part of the training data
            \item Calculate precision and recall to verify approach
        \end{itemize}
    \end{example}
\end{frame}

%------------------------------------------------
\section{Expected Results}
%------------------------------------------------

\begin{frame}{Expected Results}
    \begin{block}{Expectation}
        We expect to find at least some features of the used language that will differ between the members of opposing political parties. 
    \end{block}
    
    \begin{alertblock}{Challenges and alternative}
        \begin{itemize}
            \item Features of daily tweets might not be significant enough to differentiate.
            \item A second approach would be to limit the tweets to a single political topic.
        \end{itemize}
    \end{alertblock}
\end{frame}

%------------------------------------------------
\section{Questions}
%------------------------------------------------

\begin{frame}{Questions}
    \Huge{\centerline{Thank you for your attention!}}
\end{frame}

%----------------------------------------------------------------------------------------

\end{document}
